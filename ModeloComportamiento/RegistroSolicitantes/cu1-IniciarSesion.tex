%%%% Caso de uso %%%%

\pagebreak
\hypertarget{CU1}{}
\section{CU1 Seleccionar Pizza}

	%Resumen
	\noindent \textbf{Resumen}\\

		Para que el pizza costumer pueda acceder al módulo de seleccionar pizza, deberá ingresar desde su navegador al dominio {\textbf{www.pizzaplaneta.com.mx}} y de esta manera poder visualizar la pantalla principal del sistema.

	%Atributos
	\noindent \textbf{\\Atributos}

		\begin{itemize}

			\item \textit{Actor(es):} \hyperlink{A:Cliente}{Cliente}
			\item \textit{Precondiciones:} Que el \hyperlink{A:Cliente}{Cliente} tenga acceso a internet. 
			\item \textit{Postcondiciones:} El \hyperlink{A:Cliente}{Cliente} podrá ingresar al sistema para realizar las actividades que su rol le permite.
			\item \textit{Ciclos de vida:} ??
			\item \textit{Reglas de negocio:} ??
			\item \textit{Mensajes:} ??
			\item \textit{Viene de:} No aplica

		\end{itemize} 

	%Trayectoria principal
	\noindent \textbf{Trayectoria principal}

		\begin{enumerate}
			\item {\textbf{Actor.}} Ingresa la dirección electrónica de la pizzería en su navegador.
			
			\item \textbf{Sistema.} Muestra la pantalla \textbf{IU1 Menú de pizza} descrita en \hyperlink{IU1}{IU1 Menú de pizza}.
			
			\item \textbf{Actor.} Selecciona una de las opciones que se encuentran en la pantalla.\hyperlink{CU1:TAA}{Trayectoria alternativa A},\hyperlink{CU1:TAB}{Trayectoria alternativa B}. 
			
			\item \hypertarget{CU1:TP:P4}{\textbf{Sistema.}} Muestra la pantalla  \textbf{IU2 Personaliza pizza} descrita en \hyperlink{IU2}{IU2 Personaliza Pizza}
			
			\item \textbf{Actor.} Selecciona el tamaño y la masa que desea. \hyperlink{CU1:TAC}{Trayectoria alternativa C.}
			
			\item  \hypertarget{CU1:TP:P6}{\textbf{Actor.}} Da click en el botón \textbf{Agregar al carrito}.
			
			\item \textbf{Sistema.} Muestra la pantalla \textbf{IU1 Agregar otra}.
			
			
		\end{enumerate}	
		Fin de la trayectoria principal

	%Trayectorias alternativas
	\noindent \textbf{\\Trayectorias alternativas}

	\begin{itemize}

		%Trayectoria alternativa
		\item \hypertarget{CU1:TAA}{Trayectoria alternativa A}

			\noindent \textbf{Condición:} El actor no seleccionó una de las opciones mostradas en la pantalla, sin embargo dio click en el botón de menú en la pantalla \textbf{IU1 Inicio}.
			
			\begin{enumerate}
				\item \textbf{Actor.} Da click en el botón \textbf{Menú}.
				\item \textbf{Sistema.} Regresa al paso \hyperlink{CU1:TP:P2}{2} de la trayectoria principal
			\end{enumerate}
			
			Fin de la trayectoria alternativa
			
			%Trayectoria alternativa
			\item \hypertarget{CU1:TAB}{Trayectoria alternativa B}
			
			\noindent \textbf{Condición:} El actor no seleccionó una de las opciones mostradas en la pantalla, sin embargo dio click en el botón de Carrito \textbf{IU3 Carrito de compras}.
			
			\begin{enumerate}
				\item \textbf{Actor.} Da click en el botón \textbf{Carrito}.
				\item \textbf{Sistema.} Muestra en pantalla \hyperlink{IU3}{\textbf{IU3 Carrito de compras}}
			\end{enumerate}
			
			Fin de la trayectoria alternativa
			
		%Trayectoria alternativa
		\item \hypertarget{CU1:TAC}{Trayectoria alternativa c}
			
			\noindent \textbf{Condición:} El actor no seleccionó tamaño de pizza o tipo de masa.
			
			\begin{enumerate}
				\item \textbf{Actor.} Selecciona el botón \textbf{Agregar al carrito} sin seleccionar ninguna opción. 
				
				\item \textbf{Sistema.} Verifica la regla de negocio \textbf{RN:S2:SelecciónRequerida}.
				
				\item \textbf{Sistema.} Muestra en pantalla el mensaje \hyperlink{MSG:SeleccionarOpcion}{\textbf{MSG1-Seleccionar una opción}}. 
				
				\item \textbf{Actor.} Da click en \textbf{Aceptar}.
				
				\item \hypertarget{CU1:TAB:P4} {\textbf{Sistema.}} Regresa al paso 4 de la trayectoria principal.
	
			\end{enumerate}
			
			Fin de la trayectoria alternativa
		

	\end{itemize}

	%Puntos de extención
	\noindent \textbf{\\Puntos de extensión}\\

		%Punto de extención
		\noindent \textit{Causa de la extensión:} El actor debe seleccioanr sus datos al sistema.\\
		\textit{Región de la trayectoria:} Paso \hyperlink{CU1:TAB:P4}{4} de la trayectoria alternativa B.\\
		\textit{Extiende a:} \hyperlink{CU2.1}{CU2.1 Ingresar datos del cliente}\\

		
\hypertarget{CU2}{}
\section{CU2 Ver carrito de compras}

%Resumen
\noindent \textbf{Resumen}\\

Para que el pizza costumer pueda acceder al módulo de seleccionar pizza, deberá ingresar desde su navegador al dominio {\textbf{www.pizzaplaneta.com.mx}} y de esta manera poder visualizar la pantalla principal del sistema desde la cual dando click en Menú se llega a este caso de uso.

%Atributos
\noindent \textbf{\\Atributos}

\begin{itemize}
	
	\item \textit{Actor(es):} \hyperlink{A:Pizza Customer}{Pizza Customer}
	\item \textit{Precondiciones:} Que el \hyperlink{A:Pizza Customer}{Pizza Customer} tenga acceso a internet. 
	\item \textit{Postcondiciones:} El \hyperlink{A:Pizza Customer}{Pizza Customer} podrá gestionar sus carrito de compras.
	\item \textit{Ciclos de vida:} ?? 
	\item \textit{Reglas de negocio:} ??
	\item \textit{Mensajes:} ??
	\item \textit{Viene de:} No aplica.
	
\end{itemize} 
		
%Trayectoria principal
\noindent \textbf{Trayectoria principal}

\begin{enumerate}
	\item {\textbf{Actor.}} Ingresa la dirección electrónica de la pizzería en su navegador.
	
	\item \textbf{Sistema.} Muestra la pantalla \textbf{IU1 Menú de pizza} descrita en \hyperlink{IU1}{IU1 Inicio}.
	
	\item {\textbf{Actor.}} Da click en el botón de Carrito.
	
	\item \hypertarget{CU2:TP:P4}{\textbf{Sistema.}} Muestra la pantalla \textbf{IU3 Carrito de compras} descrita en \hyperlink{IU3}{IU3 Carrito de compras}.
	
	\item \textbf{Actor.} El actor da click en el botón  Proceder con la compra.
	\\ \hyperlink{CU2:TAA}{Trayectoria alternativa A}.
	\\ \hyperlink{CU2:TAB}{Trayectoria alternativa B}. 
	\\ \hyperlink{CU2:TAC}{Trayectoria alternativa C}. 
	
	\item \hypertarget{CU1:TP:P4}{\textbf{Sistema.}} Extiende el caso de uso \textbf{CU3 Realizar compra de la pizza}, muestra \textbf{IU4 Datos personales}.
	

\end{enumerate}	
Fin de la trayectoria principal
\\
%Trayectorias alternativas
\noindent \textbf{\\Trayectorias alternativas}

\begin{itemize}
	
	%Trayectoria alternativa
	\item \hypertarget{CU2:TAA}{Trayectoria alternativa A}
	
	\noindent \textbf{Condición:} El actor selecciono el botón de \textbf{carrito} en la pantalla \textbf{IU3 Carrito de compras}.
	
	\begin{enumerate}
		\item \textbf{Actor.} Da click en el botón \textbf{Carrito}.
		\item \textbf{Sistema.} Regresa al paso \hyperlink{CU2:TP:P4}{4} de la trayectoria principal.
	\end{enumerate}
	
	Fin de la trayectoria alternativa
	
	%Trayectoria alternativa
	\item \hypertarget{CU2:TAB}{Trayectoria alternativa B}
	
	\noindent \textbf{Condición:} El actor selecciono cualquiera de los botones \textbf{eliminar} en la pantalla \textbf{IU3 Carrito de compras}.
	
	\begin{enumerate}
		\item \textbf{Actor.} Da click en cualquiera de los botones  \textbf{Eliminar}.
		\item \textbf{Sistema.}Muestra una alerta en pantalla con dos opciones, \textbf{cancelar} y \textbf{aceptar}.
		\\ \textbf{Cancelar}, regresa al paso \hyperlink{CU2:TP:P4}{4} de la trayectoria principal.
		\\ \textbf{Aceptar} continua la trayectoria.
		\item \textbf{Sistema.}Elimina la fila de la tabla en la cual se dio el click.
		\item \textbf{Sistema.}Recarga la pagina con la tabla actualizada del carrito de compras.
	\end{enumerate}
	
	Fin de la trayectoria alternativa
	
	%Trayectoria alternativa
	\item \hypertarget{CU2:TAC}{Trayectoria alternativa C}
	
	\noindent \textbf{Condición:} El actor selecciono el botón \textbf{Menú} o el botón \textbf{Seguir comprando} en la pantalla \textbf{IU3 Carrito de compras}.
	
	\begin{enumerate}
		\item \textbf{Actor.} Da click en botón \textbf{Menú} o el botón \textbf{Seguir comprando}.
		\item \textbf{Sistema.} Muestra la pantalla \textbf{IU1 Inicio}.
	\end{enumerate}
	
	Fin de la trayectoria alternativa

\end{itemize}

\hypertarget{CU2}{}
\section{CU3 Realizar la compra}

%Resumen
\noindent \textbf{Resumen}\\

Para que el pizza costumer pueda acceder al módulo de seleccionar pizza, deberá ingresar desde su navegador al dominio {\textbf{www.pizzaplaneta.com.mx}} y de esta manera poder visualizar la pantalla principal del sistema desde la cual dando click en Menú se llega a este caso de uso.

%Atributos
\noindent \textbf{\\Atributos}

\begin{itemize}
	
	\item \textit{Actor(es):} \hyperlink{A:Pizza Customer}{Pizza Customer}
	\item \textit{Precondiciones:} Que el \hyperlink{A:Pizza Customer}{Pizza Customer} tenga acceso a internet. 
	\item \textit{Postcondiciones:} El \hyperlink{A:Pizza Customer}{Pizza Customer} podrá gestionar sus carrito de compras.
	\item \textit{Ciclos de vida:} ?? 
	\item \textit{Reglas de negocio:} ??
	\item \textit{Mensajes:} ??
	\item \textit{Viene de:} No aplica.
	
\end{itemize} 
