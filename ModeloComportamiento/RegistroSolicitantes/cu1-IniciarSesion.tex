%%%% Caso de uso %%%%

\pagebreak
\hypertarget{CU1}{}
\section{CU1 Seleccionar Pizza}

	%Resumen
	\noindent \textbf{Resumen}\\

		Para que el cliente pueda acceder al módulo de seleccionar pizza, deberá ingresar desde su navegador al dominio {\textbf{www.pizzaplaneta.com.mx}} y de esta manera poder visualizar la pantalla principal del sistema.

	%Atributos
	\noindent \textbf{\\Atributos}

		\begin{itemize}

			\item \textit{Actor(es):} \hyperlink{A:Cliente}{Cliente}
			\item \textit{Precondiciones:} Que el \hyperlink{A:Cliente}{Cliente} tenga acceso a internet. 
			\item \textit{Postcondiciones:} El \hyperlink{A:Cliente}{Cliente} podrá ingresar al sistema para realizar las actividades que su rol le permite.
			\item \textit{Ciclos de vida:} No aplica.
			\item \textit{Reglas de negocio:} ??
			\item \textit{Mensajes:} MSG1 Campos Obligatorios Faltantes y MSG2 Pizza Agregada Exitosamente.
			\item \textit{Viene de:} No aplica.

		\end{itemize} 

	%Trayectoria principal
	\noindent \textbf{Trayectoria principal}

		\begin{enumerate}
			\item {\textbf{Actor.}} Ingresa la dirección electrónica de la pizzería en su navegador.
			
			\item \hypertarget{CU1:TP:P2}{\textbf{Sistema.}} Muestra la pantalla \textbf{IU1 Menú de pizza} descrita en \hyperlink{IU1}{IU1 Menú de pizza}.
			
			\item \textbf{Actor.} Selecciona una de las opciones que se encuentran en la pantalla.\hyperlink{CU1:TAA}{Trayectoria alternativa A},\hyperlink{CU1:TAB}{Trayectoria alternativa B}. 
			
			\item \textbf{Sistema.} Muestra la pantalla  \textbf{IU2 Personaliza Pizza} descrita en \hyperlink{IU2}{IU2 Personaliza Pizza}
			
			\item \textbf{Actor.} Selecciona el tamaño, la masa y la cantidad que desea. \hyperlink{CU1:TAC}{Trayectoria alternativa C.}
			
			\item  \hypertarget{CU1:TP:P6}{\textbf{Actor.}} Da click en el botón \textbf{Agregar al carrito}.
			
			\item \hypertarget{CU1:TP:P7}{\textbf{Sistema.}} Muestra el mensaje \textbf{MSG2 Pizza Agregada Exitosamente} descrito en \hyperlink{MSG2:PizzaAgregada}{MSG2 Pizza Agregada Exitosamente} .
			
			\item \textbf{Actor.} Da click en el botón \textbf{Aceptar}. 
			
			\item Regresa al paso \hyperlink{CU1:TP:P2}{2} de la trayectoria principal. 
			
			
		\end{enumerate}	
		Fin de la trayectoria principal

	%Trayectorias alternativas
	\noindent \textbf{\\Trayectorias alternativas}

	\begin{itemize}

		%Trayectoria alternativa
		\item \hypertarget{CU1:TAA}{Trayectoria alternativa A}

			\noindent \textbf{Condición:} El actor no seleccionó una de las opciones mostradas en la pantalla, sin embargo dio click en el botón de menú en la pantalla \textbf{IU1 Menú de Pizzas}.
			
			\begin{enumerate}
				\item \textbf{Actor.} Da click en el botón \textbf{Menú}.
				\item \textbf{Sistema.} Regresa al paso \hyperlink{CU1:TP:P2}{2} de la trayectoria principal
			\end{enumerate}
			
			Fin de la trayectoria alternativa
			
			%Trayectoria alternativa
			\item \hypertarget{CU1:TAB}{Trayectoria alternativa B}
			
			\noindent \textbf{Condición:} El actor no seleccionó una de las opciones mostradas en la pantalla, sin embargo dio click en el botón de Carrito.
			
			\begin{enumerate}
				\item \textbf{Actor.} Da click en el botón \textbf{Carrito}.
				\item \textbf{Sistema.} Muestra en pantalla \hyperlink{IU3}{\textbf{IU3 Carrito de Compras}}
			\end{enumerate}
			
			Fin de la trayectoria alternativa
			
		%Trayectoria alternativa
		\item \hypertarget{CU1:TAC}{Trayectoria alternativa c}
			
			\noindent \textbf{Condición:} El actor no seleccionó tamaño de pizza o tipo de masa o cantidad deseada.
			
			\begin{enumerate}
				\item \textbf{Actor.} Selecciona el botón \textbf{Agregar al carrito} sin seleccionar ninguna opción. 
				
				\item \textbf{Sistema.} Verifica la regla de negocio \textbf{RN:S2:SelecciónRequerida}.
				
				\item \textbf{Sistema.} Muestra en pantalla el mensaje \textbf{MSG1 Campos Obligatorios Faltantes} descrito en \hyperlink{MSG1:CamposObligatorios}{\textbf{MSG1 Campos Obligatorios Faltantes}}. 
				
				\item \textbf{Actor.} Agrega los campos faltantes.
				
				\item \textbf{Sistema.} Regresa al paso \hyperlink{CU1:TP:P7}{7} de la trayectoria principal.
	
			\end{enumerate}
			
			Fin de la trayectoria alternativa
		

	\end{itemize}

	%Puntos de extención
	\noindent \textbf{\\Puntos de extensión}\\

		%Punto de extención
		\noindent \textit{Causa de la extensión:} No aplica.\\
		\textit{Región de la trayectoria:} No aplica.\\
		\textit{Extiende a:} No aplica.\\

		
	%Puntos de inclusión
	\noindent \textbf{\\Puntos de inclusión}\\
	
		%Punto de inclusión
		\noindent \textit{Causa de la inclusión:} No aplica.\\
		\textit{Región de la trayectoria:} No aplica.\\
		\textit{Incluye a:} No aplica.\\


\hypertarget{CU2}{}
\section{CU2 Ver Carrito de Compras}

%Resumen
\noindent \textbf{Resumen}\\

Para que el cliente pueda acceder al módulo de ver carrito de compras, deberá dar click al botón carrito que se encuentra en la pantalla principal del sistema.

%Atributos
\noindent \textbf{\\Atributos}

\begin{itemize}
	
	\item \textit{Actor(es):} \hyperlink{A:Cliente}{Cliente}
	\item \textit{Precondiciones:} Que el \hyperlink{A:Cliente}{Cliente} tenga acceso a internet. 
	\item \textit{Postcondiciones:} El \hyperlink{A:Cliente}{Cliente} podrá gestionar sus carrito de compras.
	\item \textit{Ciclos de vida:} No aplica. 
	\item \textit{Reglas de negocio:} ??
	\item \textit{Mensajes:} MSG3 Advertencia.
	\item \textit{Viene de:} No aplica.
	
\end{itemize} 
		
%Trayectoria principal
\noindent \textbf{Trayectoria principal}

\begin{enumerate}
	
	\item {\textbf{Actor.}} Estando dentro del sistema da click en el botón de \textbf{Carrito}.
	
	\item \hypertarget{CU2:TP:P4}{\textbf{Sistema.}} Muestra la pantalla \textbf{IU3 Carrito de Compras} descrita en \hyperlink{IU3}{IU3 Carrito de Compras}.
	
	\item \textbf{Actor.} El actor da click en el botón  \textbf{Proceder con la compra}.
	\\ \hyperlink{CU2:TAA}{Trayectoria alternativa A}.
	\\ \hyperlink{CU2:TAB}{Trayectoria alternativa B}. 
	\\ \hyperlink{CU2:TAC}{Trayectoria alternativa C}. 
	
	\item \hypertarget{CU1:TP:P4}{\textbf{Sistema.}} Extiende al caso de uso \textbf{CU3 Realizar la Compra}, muestra \textbf{IU4 Datos de Compra}.
	

\end{enumerate}	
Fin de la trayectoria principal
\\
%Trayectorias alternativas
\noindent \textbf{\\Trayectorias alternativas}

\begin{itemize}
	
	%Trayectoria alternativa
	\item \hypertarget{CU2:TAA}{Trayectoria alternativa A}
	
	\noindent \textbf{Condición:} El actor seleccionó el botón de \textbf{Carrito} en la pantalla \textbf{IU3 Carrito de Compras}.
	
	\begin{enumerate}
		\item \textbf{Actor.} Da click en el botón \textbf{Carrito}.
		\item \textbf{Sistema.} Regresa al paso \hyperlink{CU2:TP:P4}{4} de la trayectoria principal.
	\end{enumerate}
	
	Fin de la trayectoria alternativa
	
	%Trayectoria alternativa
	\item \hypertarget{CU2:TAB}{Trayectoria alternativa B}
	
	\noindent \textbf{Condición:} El actor seleccionó cualquiera de los botones \textbf{Eliminar} en la pantalla \textbf{IU3 Carrito de compras}.
	
	\begin{enumerate}
		\item \textbf{Actor.} Da click en cualquiera de los botones  \textbf{Eliminar}.
		\item \textbf{Sistema.}Muestra en pantalla el mensaje \textbf{MSG3 Advertencia} descrito en \hyperlink{MSG3:Advertencia}{\textbf{MSG3 Advertencia}} con dos opciones, \textbf{Cancelar} y \textbf{Aceptar}.
		\\ \textbf{Cancelar}: Regresa al paso \hyperlink{CU2:TP:P4}{4} de la trayectoria principal.
		\\ \textbf{Aceptar}: Continua la trayectoria.
		\item \textbf{Sistema.}Elimina la fila de la tabla en la cual se dio el click.
		\item \textbf{Sistema.}Recarga la pagina con la tabla actualizada del carrito de compras.
	\end{enumerate}
	
	Fin de la trayectoria alternativa
	
	%Trayectoria alternativa
	\item \hypertarget{CU2:TAC}{Trayectoria alternativa C}
	
	\noindent \textbf{Condición:} El actor selecciono el botón \textbf{Menú} o el botón \textbf{Seguir Comprando} en la pantalla \textbf{IU3 Carrito de compras}.
	
	\begin{enumerate}
		\item \textbf{Actor.} Da click en botón \textbf{Menú} o el botón \textbf{Seguir Comprando}.
		\item \textbf{Sistema.} Muestra la pantalla \textbf{IU1 Menú de Pizzas}.
	\end{enumerate}
	
	Fin de la trayectoria alternativa

\end{itemize}

%Puntos de extención
\noindent \textbf{\\Puntos de extensión}\\

%Punto de extención
\noindent \textit{Causa de la extensión:} El actor debe ingresar al sistema los datos necesarios para continuar con la compra.\\
\textit{Región de la trayectoria:} El CU2 extiende al CU3 en el punto \hyperlink{CU2:TP:P4}{4} de la trayectoria principal. \\
\textit{Extiende a:} \textbf{CU3 Realizar la compra}.\\


%Puntos de inclusión
\noindent \textbf{\\Puntos de inclusión}\\

%Punto de inclusión
\noindent \textit{Causa de la inclusión:} No aplica.\\
\textit{Región de la trayectoria:} No aplica.\\
\textit{Incluye a:} No aplica.\\

\hypertarget{CU3}{}
\section{CU3 Realizar la Compra}

%Resumen
\noindent \textbf{Resumen}\\

El cliente decidió proseguir con la compra por lo que ahora debe proporcionar sus datos personales para poder llevar un registro de quien compra.

%Atributos
\noindent \textbf{\\Atributos}

\begin{itemize}
	
	\item \textit{Actor(es):} \hyperlink{A:Cliente}{Cliente}
	\item \textit{Precondiciones:} Que el \hyperlink{A:Cliente}{Cliente} haya agregado al carrito de compras al menos una pizza.
	\item \textit{Postcondiciones:} El \hyperlink{A:Cliente}{Cliente} tendrá que seleccionar una forma de pago para terminar la compra.
	\item \textit{Ciclos de vida:} No aplica.
	\item \textit{Reglas de negocio:} ??
	\item \textit{Mensajes:} MSG1 Campos Obligatorios Faltantes.
	\item \textit{Viene de:} No aplica.
	
\end{itemize} 

%Trayectoria principal
\noindent \textbf{Trayectoria principal}

\begin{enumerate}
	\item {\textbf{Actor.}} El actor ingresa sus datos personales, pero en cualquier momento puede cancelar el proceso de compra. \hyperlink{CU3:TAA}{Trayectoria alternativa A}.
	\\ El actor también puede regresar a agregar más pizza. \hyperlink{CU3:TAA}{Trayectoria alternativa B}.
	\\
	
	\item {\textbf{Actor.}} Da click en \textbf{Continuar}.
	
	\item \textbf{Sistema.} Verifica que se cumpla la regla de negocio \textbf{RN Campos obligatorios} descrita en el apartado de \hyperlink{RN}{\textbf{Reglas de negocio}}, de haber un problema pasa a la \hyperlink{CU3:TAC}{Trayectoria alternativa C}.
	\\Después revisa que se cumpla la regla de negocio \textbf{RN Textos bien escritos}. \hyperlink{CU3:TAD}{Trayectoria alternativa D}
	\\Si no hay problemas continua con la trayectoria principal.
	
	\item \hypertarget{CU3:TP:P4}{\textbf{Sistema.}} Incluye el caso de uso \textbf{CU4 Seleccionar Método de Pago}.
	
	
\end{enumerate}	
Fin de la trayectoria principal
\\

\begin{itemize}
	
	%Trayectoria alternativa
	\item \hypertarget{CU3:TAA}{Trayectoria alternativa A}
	
	\noindent \textbf{Condición:} El actor quiere cancelar la compra, por lo que procede a presionar el botón \textbf{Cancelar} en la pantalla \textbf{IU3 Carrito de Compras}, si el actor presiona el botón \textbf{Carrito} hace la misma acción.
	
	\begin{enumerate}
		\item \textbf{Actor.} Da click en el botón \textbf{Carrito} o el botón \textbf{Menú}.
		\item \textbf{Sistema.} Muestra la pantalla \textbf{IU3 Carrito de Compras}.
	\end{enumerate}
	
	Fin de la trayectoria alternativa
	
	%Trayectoria alternativa
	\item \hypertarget{CU3:TAB}{Trayectoria alternativa B}
	
	\noindent \textbf{Condición:} El actor quiere agregar más pizza por lo que presiona el botón \textbf{Menú}.
	
	\begin{enumerate}
		\item \textbf{Actor.} Da click en el botón \textbf{Menú}.
		\item \textbf{Sistema.} Muestra la pantalla \textbf{IU1 Menú de Pizzas }.
	\end{enumerate}
	
	Fin de la trayectoria alternativa
	
	%Trayectoria alternativa
	\item \hypertarget{CU3:TAC}{Trayectoria alternativa C}
	
	\noindent \textbf{Condición:} No se cumplió la regla de negocio \textbf{RN Campos obligatorios}.
	
	\begin{enumerate}
		\item \textbf{Actor.} Da click en continuar sin haber completado todos los datos obligatorios.
		\item \textbf{Sistema.} Muestra el mensaje \hyperlink{MSG1:CamposObligatorios}{MSG1 Campos Obligatorios Faltantes} en la pantalla.
		\item \textbf{Actor.} Da click en \textbf{Aceptar} al mensaje.
		\item \textbf{Sistema.} Muestra la pantalla \textbf{IU4 Datos de Compra}.
	\end{enumerate}
	
	Fin de la trayectoria alternativa
	
\end{itemize}

%Puntos de extención
\noindent \textbf{\\Puntos de extensión}\\

%Punto de extención
\noindent \textit{Causa de la extensión:} No aplica.\\
\textit{Región de la trayectoria:} No aplica.\\
\textit{Extiende a:} No aplica.\\


%Puntos de inclusión
\noindent \textbf{\\Puntos de inclusión}\\

%Punto de inclusión
\noindent \textit{Causa de la inclusión:} El cliente debe seleccionar un método de pago para poder continuar con su compra.\\
\textit{Región de la trayectoria:} El CU3 incluye al CU4 en el punto \hyperlink{CU3:TP:P4}{4} de la trayectoria principal. \\
\textit{Incluye a:} \textbf{CU4 Seleccionar Método de Pago}.\\

\hypertarget{CU4}{}
\section{CU4 Seleccionar Método de Pago}

%Resumen
\noindent \textbf{Resumen}\\

El cliente quiere finalizar la compra por lo que debe seleccionar un método de pago para concluir con el proceso.

%Atributos
\noindent \textbf{\\Atributos}

\begin{itemize}
	
	\item \textit{Actor(es):} \hyperlink{A:Cliente}{Cliente}
	\item \textit{Precondiciones:} Que el \hyperlink{A:Cliente}{Cliente} haya registrado exitosamente los datos personales.
	\item \textit{Postcondiciones:} El \hyperlink{A:Cliente}{Cliente} obtendrá un comprobante de compra con el cual puede recibir su pizza.
	\item \textit{Ciclos de vida:} No aplica.
	\item \textit{Reglas de negocio:} ??
	\item \textit{Mensajes:} MSG1 Campos Obligatorios Faltantes y MSG4 Formato.
	\item \textit{Viene de:} \textbf{CU3 Realizar la Compra}.
	
\end{itemize} 

%Trayectoria principal
\noindent \textbf{Trayectoria principal}

\begin{enumerate}
	\item {\textbf{Actor.}} El actor selecciona un método de pago.
	\\ \textbf{Tarjeta}. \hyperlink{TAA:CU4}{Trayectoria alternativa A.} 
	\\ \textbf{Paypal}. \hyperlink{TAB:CU4}{Trayectoria alternativa B.}
	\\ \textbf{Efectivo}. \hyperlink{TAC:CU4}{Trayectoria alternativa C.}
	\\ En cualquier momento puede cancelar el proceso de compra. \hyperlink{TAD:CU3}{Trayectoria alternativa D}.
	\\ En cualquier momento puede regresar a cambiar un dato. \hyperlink{TAE:CU3}{Trayectoria alternativa E}.
	\\ El actor también puede regresar a agregar más pizza. \hyperlink{TAF:CU3}{Trayectoria alternativa F}.

	
	\item {\textbf{Actor.}} \hypertarget{p5}{}Da click en continuar.
	
	\item \textbf{Sistema.} \hypertarget{p6}{}Muestra la pantalla \textbf{IU6 Compra Exitosa}.
		
\end{enumerate}	
Fin de la trayectoria principal
\\

\begin{itemize}
	
	%Trayectoria alternativa
	\item \hypertarget{TAA:CU4}{Trayectoria alternativa A}
	
	\noindent \textbf{Condición:} El actor selecciona la opción de pago con tarjeta.
	
	\begin{enumerate}
		\item \textbf{Actor.} Da click en el checkbox de pago con \textbf{Tarjeta}.
		\item \textbf{Actor.} Da click en el botón \textbf{Continuar}.
		\item \textbf{Sistema.} Verifica que se cumpla la regla de negocio \textbf{RN Campos obligatorios} descrita en el apartado de \hyperlink{RN}{\textbf{Reglas de negocio}}, de haber un problema pasa a la \hyperlink{TAC:CU3}{Trayectoria alternativa G}.
		\\Después revisa que se cumpla la regla de negocio \textbf{RN Textos bien escritos}. \hyperlink{TAD:CU3}{Trayectoria alternativa H}
		\\Si no hay problemas continua con la trayectoria alternativa.
		\item \textbf{Sistema.} Regresa al paso \hyperlink{p5}{\textbf{5}} de la trayectoria principal.
		
	\end{enumerate}
	
	Fin de la trayectoria alternativa
	
	%Trayectoria alternativa
	\item \hypertarget{TAB:CU4}{Trayectoria alternativa B}
	
	\noindent \textbf{Condición:} El actor selecciona la opción de pago con tarjeta.
	
	\begin{enumerate}
		\item \textbf{Actor.} Da click en el checkbox de pago con \textbf{Tarjeta}.
		\item \textbf{Sistema.} Implementa el uso de la API de paypal  \textbf{IU1 Menú}.
		\\ Al terminar el proceso regresa al paso \hyperlink{p6}{\textbf{6}} de la trayectoria principal.
	\end{enumerate}
	
	Fin de la trayectoria alternativa
	
	%Trayectoria alternativa
	\item \hypertarget{TAC:CU4}{Trayectoria alternativa C}
	
	\noindent \textbf{Condición:} El actor selecciona la opción de pago con efectivo.
	
	\begin{enumerate}
		\item \textbf{Actor.}  Da click en el checkbox de pago en \textbf{Efectivo}.
		\item \textbf{Actor.} Da click en continuar. 
		\item \textbf{Sistema.} Regresa al paso \hyperlink{p6}{\textbf{6}} de la trayectoria principal.
	\end{enumerate}
	
	Fin de la trayectoria alternativa
	
		%Trayectoria alternativa
	\item \hypertarget{TAD:CU4}{Trayectoria alternativa D}
	
	\noindent \textbf{Condición:} El actor quiere cancelar la compra, por lo que procede a presionar el botón \textbf{Cancelar} en la pantalla \textbf{IU3 Carrito de Compras}, si el actor presiona el botón \textbf{Carrito} hace la misma acción.
	
	\begin{enumerate}
		\item \textbf{Actor.} Da click en el botón \textbf{Carrito} o el botón \textbf{Menú}.
		\item \textbf{Sistema.} Muestra la pantalla \textbf{IU3 Carrito de Compras}.
	\end{enumerate}
	
	Fin de la trayectoria alternativa
	
			%Trayectoria alternativa
	\item \hypertarget{TAE:CU4}{Trayectoria alternativa E}
	
	\noindent \textbf{Condición:} El actor quiere cambiar un dato de la pantalla \textbf{UI4 Datos de Compra}.
	
	\begin{enumerate}
		\item \textbf{Actor.} Da click en el botón \textbf{Regresar}.
		\item \textbf{Sistema.} Muestra la pantalla \textbf{IU3 Carrito de Compras}.
	\end{enumerate}
	
	Fin de la trayectoria alternativa
	
				%Trayectoria alternativa
	\item \hypertarget{TAF:CU4}{Trayectoria alternativa F}
	
	\noindent \textbf{Condición:} El actor quiere agregar más pizza en su orden.
	
	\begin{enumerate}
		\item \textbf{Actor.} Da click en el botón \textbf{Menú}.
		\item \textbf{Sistema.} Muestra la pantalla \textbf{IU1 Menú de Pizzas}.
	\end{enumerate}
	
	Fin de la trayectoria alternativa
	
					%Trayectoria alternativa
	\item \hypertarget{TAG:CU4}{Trayectoria alternativa G}
	
		\noindent \textbf{Condición:} No se cumplió la regla de negocio \textbf{RN Campos obligatorios}.
	
	\begin{enumerate}
		\item \textbf{Actor.} Da click en continuar sin haber completado todos los datos obligatorios.
		\item \textbf{Sistema.} Muestra el mensaje \hyperlink{MSG1:CamposObligatorios}{MSG1 Campos Obligatorios Faltantes} en la pantalla.
		\item \textbf{Actor.} Da click en \textbf{Aceptar} al mensaje.
		\item \textbf{Sistema.} Muestra la pantalla \textbf{IU5 Método de Pago}.
	\end{enumerate}
	
	Fin de la trayectoria alternativa
	
		\item \hypertarget{TAH:CU4}{Trayectoria alternativa H}
	
	\noindent \textbf{Condición:} No se cumplió la regla de negocio \textbf{RN Textos bien escritos}.
	
	\begin{enumerate}
		\item \textbf{Actor.} Da click en continuar habiendo escrito algo de manera incorrecta.
		\\\textbf{Ej.} Escribir un numero en un campo de texto.
		\item \textbf{Sistema.} Muestra el mensaje \hyperlink{MSG4:Formato}{MSG4 Formato} en la pantalla.
		\item \textbf{Actor.} Da click en aceptar al mensaje.
		\item \textbf{Sistema.} Muestra la pantalla \textbf{IU5 Método de pago}.
	\end{enumerate}
	
	Fin de la trayectoria alternativa
	
\end{itemize}

	%Puntos de extención
\noindent \textbf{\\Puntos de extensión}\\

%Punto de extención
\noindent \textit{Causa de la extensión:} No aplica.\\
\textit{Región de la trayectoria:} No aplica.\\
\textit{Extiende a:} No aplica.\\


%Puntos de inclusión
\noindent \textbf{\\Puntos de inclusión}\\

%Punto de inclusión
\noindent \textit{Causa de la inclusión:} No aplica.\\
\textit{Región de la trayectoria:} No aplica.\\
\textit{Incluye a:} No aplica.\\