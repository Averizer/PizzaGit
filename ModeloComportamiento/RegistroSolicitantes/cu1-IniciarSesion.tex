%%%% Caso de uso %%%%

\pagebreak
\hypertarget{CU1}{}
\section{CU1 Seleccionar Pizza}

	%Resumen
	\noindent \textbf{Resumen}\\

		Para que el pizza costumer pueda acceder al módulo de seleccionar pizza, deberá ingresar desde su navegador al dominio {\textbf{www.pizzaplaneta.com.mx}} y de esta manera poder visualizar la pantalla principal del sistema.

	%Atributos
	\noindent \textbf{\\Atributos}

		\begin{itemize}

			\item \textit{Actor(es):} \hyperlink{A:Pizza Costumer}{Pizza Costumer}
			\item \textit{Precondiciones:} Que el \hyperlink{A:Pizza Costumer}{Pizza Costumer} tenga acceso a internet. 
			\item \textit{Postcondiciones:} El \hyperlink{A:Pizza Costumer}{Pizza Costumer} podrá ingresar al sistema para realizar las actividades que su rol le permite.
			\item \textit{Ciclos de vida:} ??
			\item \textit{Reglas de negocio:} ??
			\item \textit{Mensajes:} ??
			\item \textit{Viene de:} No aplica

		\end{itemize} 

	%Trayectoria principal
	\noindent \textbf{Trayectoria principal}

		\begin{enumerate}
			\item {\textbf{Actor.}} Ingresa la dirección electrónica de la pizzería en su navegador.
			
			\item \textbf{Sistema.} Muestra la pantalla \textbf{IU1 Menú de pizza} descrita en \hyperlink{IU1}{IU1 Inicio}.
			
			\item \textbf{Actor.} Selecciona una de las opciones que se encuentran en la pantalla.\hyperlink{CU1:TAA}{Trayectoria alternativa A},\hyperlink{CU1:TAB}{Trayectoria alternativa B}. 
			
			\item \hypertarget{CU1:TP:P4}{\textbf{Sistema.}} Muestra la pantalla  \textbf{IU2 Personalizar tu pizza} descrita en \hyperlink{IU2}{IU2 Tamaño y masa}
			
			\item \textbf{Actor.} Selecciona el tamaño y la masa que desea. \hyperlink{CU1:TAC}{Trayectoria alternativa C.}
			
			\item  \hypertarget{CU1:TP:P6}{\textbf{Actor.}} Da click en el botón \textbf{Agregar al carrito}.
			
			\item \textbf{Sistema.} Muestra la pantalla \textbf{IU1 Agregar otra}.
			
			
		\end{enumerate}	
		Fin de la trayectoria principal

	%Trayectorias alternativas
	\noindent \textbf{\\Trayectorias alternativas}

	\begin{itemize}

		%Trayectoria alternativa
		\item \hypertarget{CU1:TAA}{Trayectoria alternativa A}

			\noindent \textbf{Condición:} El actor no seleccionó una de las opciones mostradas en la pantalla, sin embargo dio click en el botón de menú en la pantalla \textbf{IU1 Inicio}.
			
			\begin{enumerate}
				\item \textbf{Actor.} Da click en el botón \textbf{Menú}.
				\item \textbf{Sistema.} Regresa al paso \hyperlink{CU1:TP:P2}{2} de la trayectoria principal
			\end{enumerate}
			
			Fin de la trayectoria alternativa
			
			%Trayectoria alternativa
			\item \hypertarget{CU1:TAB}{Trayectoria alternativa B}
			
			\noindent \textbf{Condición:} El actor no seleccionó una de las opciones mostradas en la pantalla, sin embargo dio click en el botón de Carrito \textbf{IU3 Carrito de compras}.
			
			\begin{enumerate}
				\item \textbf{Actor.} Da click en el botón \textbf{Carrito}.
				\item \textbf{Sistema.} Muestra en pantalla \hyperlink{IU3}{\textbf{IU3 Carrito de compras}}
			\end{enumerate}
			
			Fin de la trayectoria alternativa
			
		%Trayectoria alternativa
		\item \hypertarget{CU1:TAC}{Trayectoria alternativa c}
			
			\noindent \textbf{Condición:} El actor no seleccionó tamaño de pizza o tipo de masa.
			
			\begin{enumerate}
				\item \textbf{Actor.} Selecciona el botón \textbf{Agregar al carrito} sin seleccionar ninguna opción. 
				
				\item \textbf{Sistema.} Verifica la regla de negocio \textbf{RN:S2:SelecciónRequerida}.
				
				\item \textbf{Sistema.} Muestra en pantalla el mensaje \hyperlink{MSG:SeleccionarOpcion}{\textbf{MSG1-Seleccionar una opción}}. 
				
				\item \textbf{Actor.} Da click en \textbf{Aceptar}.
				
				\item \hypertarget{CU1:TAB:P4} {\textbf{Sistema.}} Regresa al paso 4 de la trayectoria principal.
	
			\end{enumerate}
			
			Fin de la trayectoria alternativa
		

	\end{itemize}

	%Puntos de extención
	\noindent \textbf{\\Puntos de extensión}\\

		%Punto de extención
		\noindent \textit{Causa de la extensión:} El actor debe seleccioanr sus datos al sistema.\\
		\textit{Región de la trayectoria:} Paso \hyperlink{CU1:TAB:P4}{4} de la trayectoria alternativa B.\\
		\textit{Extiende a:} \hyperlink{CU2.1}{CU2.1 Ingresar datos del cliente}\\

		
\section{CU4 Seleccionar tipo de entrega}

%Resumen
\noindent \textbf{Resumen}\\

Para acceder al caso de uso ingresar datos, es necesario que usuario haya seleccionado una pizza y haya dado click al botón de proceder a pagar. En este caso de uso se toma la decisión de entrega a domicilio o recoger en sucursal.

%Atributos
\noindent \textbf{\\Atributos}

\begin{itemize}
	
	\item \textit{Actor(es):} \hyperlink{A:Pizza Costumer}{Pizza Costumer}
	\item \textit{Precondiciones:} Que el \hyperlink{A:Pizza Costumer}{Pizza Costumer} haya terminado de seleccionar lo que va a consumir y prosiga con la forma de pago. 
	\item \textit{Postcondiciones:} El \hyperlink{A:Pizza Costumer}{Pizza Costumer} podrá Seleccionar la forma de pago.
	\item \textit{Ciclos de vida:} ??
	\item \textit{Reglas de negocio:} ??
	\item \textit{Mensajes:} ??
	\item \textit{Viene de:} ??
	
\end{itemize} 

%Trayectoria principal
\noindent \textbf{Trayectoria principal}

\begin{enumerate}
	\item {\textbf{Actor.}} El usuario da click en el botón ``En tienda". \hypertarget{CU4:TAB}{Trayectoria alternativa A.}\newline
	El usuario da click en el botón ``Domicilio". \hypertarget{CU4:TAB}{Trayectoria alternativa B.} 
	\item \textbf{Sistema.} Muestra la pantalla \textbf{IU4 Ingesar datos personales} trayectoria alternativa A o bien \textbf{IU3 Ingesar dirección} trayectoria alternativa B.
	\item \textbf{Actor.} Ingresa los datos requeridos, si no son validos pasa a trayectoria alternativa C.
	\item \hypertarget{CU1:TP:P4}{\textbf{Sistema.}} Muestra la pantalla  \textbf{IU5 Selecciona la forma de pago}.
	
\end{enumerate}

Fin de la trayectoria principal

%Trayectorias alternativas
\noindent \textbf{\\Trayectorias alternativas}

\begin{itemize}
	
	%Trayectoria alternativa
	\item \hypertarget{CU4:TAA}{Trayectoria alternativa A}
	
	\noindent \textbf{Condición:} El actor dió click en el botón de Recoger en tienda en \textbf{IU2 Seleccionar donde recibir}.
	
	\begin{enumerate}
		\item \textbf{Sistema.} Muestra \textbf{IU3 Datos personales}.
		\item \textbf{Actor.} Ingresa datos personales y da click en metodo de pago.
		\item \textbf{Sistema.} Muestra \textbf{IU5 Forma de pago}.
	\end{enumerate}
	
	Fin de la trayectoria alternativa
	
	%Trayectoria alternativa
	\item \hypertarget{CU1:TAB}{Trayectoria alternativa B}
	
	\noindent \textbf{Condición:} El actor da click en el boton de envio a domicilio en \textbf{IU2 Seleccionar donde recibir}.

\begin{enumerate}
	\item \textbf{Sistema.} Muestra \textbf{IU4 Dirección}.
	\item \textbf{Actor.} Ingresa datos personales y da click en siguiente.
	\item \textbf{Sistema.} Muestra \textbf{IU5 Forma de pago}.
	\item \textbf{Actor.} Ingresa datos personales y da click en metodo de pago.
	\item \textbf{Sistema.} Muestra \textbf{IU5 Forma de pago}.
\end{enumerate}

	Fin de la trayectoria alternativa
	
	%Trayectoria alternativa
	\item \hypertarget{CU1:TAC}{Trayectoria alternativa C}
	
	\noindent \textbf{Condición:} El actor no ingresa correctamente los datos.
	
	\begin{enumerate}
		
		\item \textbf{Sistema.} El sistema pasa recarga la página mostrando el mensaje \textbf{MSG1 Datos no validos}
		
	\end{enumerate}
	
	Fin de la trayectoria alternativa
\end{itemize}	
	

%Puntos de extención
\noindent \textbf{\\Puntos de extensión}\\

\noindent \textit{Causa de la extensión:} El actor debe ingresar sus datos al sistema.\\
\textit{Región de la trayectoria:} Paso \hyperlink{CU1:TAB:P4}{4} de la trayectoria alternativa B.\\
\textit{Extiende a:} \hyperlink{CU2.1}{CU2.1 Ingresar datos del cliente}\\

