%%%% Pantalla(s) asociadas a caso de uso %%%%

\pagebreak
\hypertarget{IU1}{}
\section{IU1 Iniciar sesión}

	%Descripción de la pantalla
	\noindent \textbf{Descripción de pantalla(s)}\\

		La pantalla mostrada en la figura \ref{IU1} tiene como objetivo permitir a los solicitante con cuenta activa el ingreso al sistema. Se tienen como campos de entrada: \textbf{Usuario} y \textbf{Contraseña}, ambos obligatorios. 
		
		Una vez que el solicitante ingresa la información requerida, deberá presionar el botón \textbf{Aceptar} para que el sistema permita el inicio de sesión y si los datos proporcionados son correctos, se mostrará la pantalla correspondiente a la figura \ref{IU1-A}, la cual describe la bienvenida en el sistema para el solicitante y da entrada a cada una de las funcionalidades a las que el usuario tiene derecho.

		%Pantalla del sistema
		\begin{figure}[h]

			\begin{center}				

				\includegraphics[scale=0.35]{./imagenes/IUs/RegistroSolicitantes/iu1-IniciarSesion/IU1-IniciarSesion.jpg}
				\caption{IU1 Iniciar sesión}
				\label{IU1}

			\end{center}
				
		\end{figure}

		%Pantalla del sistema
		\begin{figure}[h]

			\begin{center}
			
				\includegraphics[scale=0.35]{./imagenes/IUs/RegistroSolicitantes/iu1-IniciarSesion/IU1-A-BienvenidaSolicitante.jpg}
				\caption{IU1-A Bienvenida para solicitante}
				\label{IU1-A}

			\end{center} 				

		\end{figure}

		\pagebreak
		Si el solicitante no se encuentra registrado en el sistema y desea obtener una cuenta de usuario, en la pantalla \ref{IU1} se encuentra el link \textbf{Registrarse}, mediante el cual se tendrá acceso a la funcionalidad que permite conseguir una cuenta de usuario si las condiciones en el sistema lo permiten.

		Cuando el solicitante haya olvidado su contraseña de usuario para iniciar sesión y desee obtener una nueva, en la pantalla \ref{IU1} se encuentra el link \textbf{Recuperar contraseña}, mediante el cual se tendrá acceso a la funcionalidad que permitirá obtener una nueva contraseña.

		El caso de uso \hyperlink{CU1}{CU1 Iniciar sesión}, describe de forma detallada el comportamiento asociado.

	%Acciones asociadas
	\noindent \textbf{\\Acciones en pantalla}

		\begin{itemize}

			\item \includegraphics[scale=0.03]{imagenes/iconografia/aceptar.jpg}: Dirige a la pantalla mostrada en  la figura \ref{IU1-A}
			\item \textit{Registrarse:} Dirige a la pantalla \hyperlink{IU1.1}{IU1.1 Registrar solicitante}
			\item \textit{Recuperar contraseña:} Dirige a la pantalla \hyperlink{IU1.2}{IU1.2 Recuperar contraseña}

		\end{itemize}


