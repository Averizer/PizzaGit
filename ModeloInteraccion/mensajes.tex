%%%% Catálogo de mensajes %%%%

\pagebreak
\section{Mensajes}

	%Definición de mensaje
	\hypertarget{MSG1:CamposObligatorios}{}
	\subsection{MSG1 Campos Obligatorios Faltantes}

		\begin{itemize}

			\item \textit{Tipo:} Error.

			\item\textit{Ubicación:}  Debajo de cada campo marcado como obligatorio donde no se proporcionó información.

			\item \textit{Objetivo:} Notificar al cliente la omisión de uno o más campos marcados como obligatorios.

			\item \textit{Redacción:} El campo \textbf{CAMPO} es obligatorio.

			\item \textit{Parámetros:} \textbf{CAMPO} es el nombre del campo obligatorio que ha sido omitido.

			\item \textit{Ejemplo:} Suponga que el campo \textbf{Nombre} es obligatorio y se omite en el ingreso de información. El sistema deberá mostrar debajo de la petición de dicho campo el mensaje: \textbf{El campo Nombre es obligatorio}.

			\item \textit{Referenciado por:} \hyperlink{CU1}{CU1 Seleccionar Pizza}, \hyperlink{CU3}{CU3 Realizar la Compra} y \hyperlink{CU4}{CU4 Seleccionar Método de Pago}.

		\end{itemize}
	
\hypertarget{MSG2:PizzaAgregada}{}
\subsection{MSG2 Pizza Agregada Exitosamente}

\begin{itemize}
	
	\item \textit{Tipo:} Informativo.
	
	\item\textit{Ubicación:}  En el momento en el que una pizza quede registrada en el carrito de compras, el sistema lanzará una ventana modal.
	
	\item \textit{Objetivo:} Notificar al cliente de que su pizza se agregó correctamente.
	
	\item \textit{Redacción:} El elemento \textbf{ELEMENTO} se agregó correctamente.
	
	\item \textit{Parámetros:} \textbf{ELEMENTO} es el nombre del objeto que se agregó al carrito de compras.
	
	\item \textit{Ejemplo:} Suponga que el cliente agregó una \textbf{Pizza} al carrito de compras. El sistema deberá mostrar mediante una ventana modal el siguiente mensaje: \textbf{Pizza agregada correctamente}.
	
	\item \textit{Referenciado por:} \hyperlink{CU1}{CU1 Seleccionar Pizza}.
	
\end{itemize}

\hypertarget{MSG3:Advertencia}{}
\subsection{MSG3 Advertencia}

\begin{itemize}
	
	\item \textit{Tipo:} Decisión.
	
	\item\textit{Ubicación:}  En el momento en el que se desee eliminar alguna pizza del carrito de compras, el sistema lanzará una ventana modal.
	
	\item \textit{Objetivo:} Notificar al cliente de que está a punto de eliminar uno de los elementos de su carrito de compras.
	
	\item \textit{Redacción:} Seguro deseas eliminar el elemento \textbf{ELEMENTO}.
	
	\item \textit{Parámetros:} \textbf{ELEMENTO} es el nombre del objeto que será eliminado del carrito de compras.
	
	\item \textit{Ejemplo:} Suponga que el cliente desea eliminar una \textbf{Pizza} dexl carrito de compras. El sistema deberá mostrar mediante una ventana modal el siguiente mensaje: \textbf{¿Seguro deseas quitar la pizza?}.
	
	\item \textit{Referenciado por:} \hyperlink{CU2}{CU2 Ver Carrito de Compras}
	
\end{itemize}

\hypertarget{MSG4:Formato}{}
\subsection{MSG4 Formato}

\begin{itemize}
	
	\item \textit{Tipo:} Error.
	
	\item\textit{Ubicación:}  En el momento en el que el formato de algún dato sea incorrecto.
	
	\item \textit{Objetivo:} Notificar al cliente de que el formato que ingresó no es correcto.
	
	\item \textit{Redacción:} El campo \textbf{CAMPO} tiene un formato que no es válido.
	
	\item \textit{Parámetros:} \textbf{CAMPO} es el nombre del campo que tiene un formato erróneo.
	
	\item \textit{Ejemplo:} Suponga que el cliente ingresó un caracter especial en el campo \textbf{Teléfono}. El sistema deberá mostrar debajo de la petición de dicho campo el mensaje: \textbf{El campo Teléfono tiene un formato inválido}.
	
	\item \textit{Referenciado por:} \hyperlink{CU4}{CU4 Seleccionar Método de Pago}.
	
\end{itemize}
