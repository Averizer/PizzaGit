%%%% Catálogo de reglas de negocio del sistema%%%%

\
\section{Reglas de negocio del sistema}

	%Regla del sistema
	\hypertarget{RN:S1:InfoRequerida}{}
	\subsection{RN-S1 Información requerida}
		
		\begin{itemize}

			\item \textit{Tipo:} Restricción

			\item \textit{Descripción:} Los campos de entrada que son marcados como requeridos con el caracter \textbf{*} antes del inicio del nombre de dichos campos, no se deben omitir.

			\item \textit{Ejemplo:} Si el campo de entrada \textbf{Nombre} es obligatorio, debe aparecer marcado como: \textbf{* Usuario}, lo cual indica que no puede ser omitido como dato de entrada.

			\item \textit{Referenciado por:} \hyperlink{CU1}{CU1 Seleccionar Pizza}, \hyperlink{CU3}{Realizar Compra}, \hyperlink{CU4}{CU4 Seleccionar Método de pago}.

		\end{itemize}
	
	\hypertarget{RN:S2:InfoValida}{}
	\subsection{RN-S2 Información válida}
	
		\begin{itemize}
		
		\item \textit{Tipo:} Restricción
		
		\item \textit{Descripción:} Los campos de entrada deben ser llenado exclusivamente con caracteres relacionado con lo solicitado.
		
		\item \textit{Ejemplo:} El campo de entrada \textbf{Teléfono} es destinado a incluir únicamente números, de lo contrario el sistema debe avisar que la información es incorrecta.
		
		\item \textit{Referenciado por:} \hyperlink{CU3}{Realizar Compra}, \hyperlink{CU4}{CU4 Seleccionar Método de pago}.
		
		
	\end{itemize}

	\hypertarget{RN:S3:ProcederCompra}{}
\subsection{RN-S3 Proceder a la compra}

\begin{itemize}
	
	\item \textit{Tipo:} Derivación
	
	\item \textit{Descripción:} El botón \textbf{Proceder a la compra} de la pantalla \hyperlink{IU3}{\textbf{IU3 Carrito de compra}} debe estar habilitado sólo si se tiene al menos un artículo en el carrito de compra.
	
	\item \textit{Ejemplo:} El cliente presiona el botón de \textbf{carrito} sin haber seleccionado al menos una pizza, por lo tanto no puede proceder a la compra.
	
	\item \textit{Referenciado por:} \hyperlink{CU2}{CU2 Ver Carrito de Compras}
	
\end{itemize}
	
	\hypertarget{RN:S4:TipoEntrega}{}
	\subsection{RN-S4 Tipo de entrega}
	
	\begin{itemize}
		
		\item \textit{Tipo:} Derivación
		
		\item \textit{Descripción:} El cliente quiere recibir la pizza en su casa por lo que selecciona la opción de \textbf{Tipo de entrega: Domicilio}, entonces los campos para entrega de domicilio se activarán.
		
		\item \textit{Ejemplo:} El cliente quiere recoger su pizza en la tienda, entonces los campos para introducir el domicilio no se mostraran en pantalla.
		
		\item \textit{Referenciado por:} No aplica.
		
	\end{itemize}
	
	\hypertarget{RN:S5:PDF}{}
\subsection{RN-S5 PDF}

\begin{itemize}
	
	\item \textit{Tipo:} Respuesta
	
	\item \textit{Descripción:} El cliente terminó su compra habiendo ya realizado su pago, por lo que se genera un PDF como confirmación de la orden.
	
	\item \textit{Ejemplo:} El cliente compra 1 pizza y termina de pagar, la siguiente pantalla muestra se mostrara un PDF para comprobar que la compra se realizó con éxito.
	
	\item \textit{Referenciado por:} No aplica.
	
\end{itemize}

	\hypertarget{RN:S6:BD}{}
\subsection{RN-S6 Base de datos}

\begin{itemize}
	
	\item \textit{Tipo:} Derivación 
	
	\item \textit{Descripción:} El cliente terminó su compra habiendo ya realizado su pago, por lo que se genera una consulta en la Base de datos donde se guarda la información de la compra.
	
	\item \textit{Ejemplo:} El cliente compra 1 pizza y termina de pagar, La base de datos debe guardar la información de la compra.
	
	\item \textit{Referenciado por:} No aplica.
	
\end{itemize}